% =============================================================================
% TEMPLATE DE ARTIGO PARA O SIINTEC - VERSÃO 2.1 (IEEE Conference style)
% Compile com pdfLaTeX.
% =============================================================================

\documentclass{siintec}

% --- PACOTES OPCIONAIS ---
\usepackage[brazil]{babel}
\usepackage{lipsum} % Para texto de exemplo (remover no artigo final)

% --- INFORMAÇÕES DO ARTIGO (ANTES do \begin{document}) ---
\title{Template for Preparing an Article}

\author{
  Author One$^{1}$, Author Two$^{2}$, Author Three*$^{3}$ \\
  $^{1}$Organization Name, Department Name, City, State, Country \\
  $^{2}$Organization Name, Department Name, City, State, Country \\
  $^{3}$Organization Name, Department Name, City, State, Country \\
  *Corresponding author: author3@email
}

\date{\selectlanguage{brazil}\today}

\begin{document}

\maketitle

% --- TÍTULO, AUTORES, RESUMO E PALAVRAS-CHAVE (EM UMA COLUNA) ---
\begin{abstract}
This document gives formatting instructions for authors preparing papers for publication in the SIINTEC. Authors are encouraged to prepare manuscripts directly using this template. Use Times New Roman, 10 (Font Size), Bold, and >1800 to 2000 characters with spaces.
\end{abstract}

\begin{keywords}
Times New Roman. 10 (Font Size). Bold. Formatting. SIINTEC.
\end{keywords}

\begin{abbreviations}
SIINTEC, Symposium on Innovation and Technology. PDF, Portable Document Format.
\end{abbreviations}

% --- INÍCIO DO LAYOUT DE DUAS COLUNAS PARA O CORPO DO ARTIGO ---
\doublespacing

\section{Ease of Use}
An easy way to comply with the paper formatting requirements is to use this document as a template and simply type your text into it~\cite{fogg2003}. The template is used to format your paper and style the text~\cite{hirsh2002}. All margins, column widths, line spaces, and text fonts are prescribed; please do not alter them.

\lipsum[1]

\section{Page Layout}
Your paper must use a page size corresponding to A4 which is 210 mm wide and 297 mm long~\cite{hirsh2002}. The margins are set as follows: top= 15 mm, bottom= 15 mm, right=17.5 mm, left = 20 mm~\cite{hirsh2002}.

\lipsum[2-4]

% --- SEÇÕES FINAIS (SEM NUMERAÇÃO) ---
\section*{Acknowledgement}
The heading of the Acknowledgment section and the References section must not be numbered.

% --- REFERÊNCIAS ---
\bibliographystyle{plain}
\bibliography{refs}

\end{document}