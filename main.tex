% =============================================================================
% TEMPLATE DE ARTIGO PARA O SIINTEC
% Use este arquivo para escrever seu conteúdo.
% Compile com pdfLaTeX.
% =============================================================================

% Carrega a classe de formatação do SIINTEC.
% O arquivo 'siintec.cls' deve estar na mesma pasta.
\documentclass{siintec}

% --- PACOTES OPCIONAIS (adicionar se necessário) ---
\usepackage[brazil]{babel} % Para hifenização em português
\usepackage{lipsum} % Para gerar texto de exemplo (remover no artigo final)

% --- INFORMAÇÕES DO ARTIGO (FRONT MATTER) ---
\title{Template para a Preparação de um Artigo para o SIINTEC} % [cite: 26]

% Use \author para cada autor e \thanks para as afiliações numéricas [cite: 28]
\author{
  Autor Um\thanks{Nome da Organização, Nome do Departamento, Cidade, Estado, País} \and
  Autor Dois\thanks{Nome da Organização, Nome do Departamento, Cidade, Estado, País} \and
  Autor Três\thanks{Nome da Organização, Nome do Departamento, Cidade, Estado, País}
}

% E-mail obrigatório do autor correspondente [cite: 6, 29]
\correspondingauthor{Corresponding author: autor3@email}


% --- INÍCIO DO DOCUMENTO ---
\begin{document}

% Gera o título, autores e afiliações
\maketitle

% --- RESUMO E PALAVRAS-CHAVE ---
\begin{abstract}
Este documento fornece instruções de formatação para autores que preparam artigos para publicação no SIINTEC. [cite: 7] Os autores são encorajados a preparar os manuscritos diretamente usando este template. [cite: 8] O resumo deve ter entre 1800 e 2000 caracteres com espaços. [cite: 9] O corpo do resumo é em Times New Roman, tamanho 10, negrito. [cite: 9]
\end{abstract}

\begin{keywords}
Template. LaTeX. SIINTEC. Artigo. Exemplo. % Mínimo 3, máximo 5 palavras, separadas por ponto [cite: 10]
\end{keywords}


% --- CORPO DO ARTIGO ---
% O texto principal deve ter espaçamento duplo [cite: 32] e fonte 12pt.
\doublespacing

\section{Facilidade de Uso}
Uma maneira fácil de cumprir os requisitos de formatação do artigo é usar este documento como um modelo e simplesmente digitar seu texto nele. [cite: 18] O modelo é usado para formatar seu artigo e estilizar o texto. [cite: 19] Todas as margens, larguras de coluna, espaços de linha e fontes de texto são predefinidas; por favor, não as altere. [cite: 20]

\lipsum[1] % Texto de exemplo. Remova \lipsum no seu artigo.

\section{Estrutura e Formatação}

\subsection{Layout da Página}
Seu artigo deve usar o tamanho de página A4. [cite: 22] As margens estão definidas na classe do documento. [cite: 23] O formato é de duas colunas. [cite: 24]
\lipsum[2]

\subsubsection{Equações}
As equações devem ser centralizadas com os números alinhados à direita. [cite: 39]
\begin{equation}
E = mc^2 \label{eq:einstein}
\end{equation}
A Equação \ref{eq:einstein} é um exemplo de formatação. Se usar o MathType, formate as equações como Times + Symbol ou Calibri Math 10. [cite: 40]

\subsection{Figuras e Tabelas}
As tabelas e figuras devem ser inseridas próximas à sua primeira menção no texto. [cite: 44]

% Exemplo de Tabela [cite: 61, 62, 63]
\begin{table}[h]
\caption{Fontes recomendadas para o artigo.} % Legenda acima da tabela.
\label{tab:fonts}
\centering
\begin{tabular}{ll}
\hline
\textbf{Tamanho} & \textbf{Uso} \\
\hline
12pt & Texto principal, Títulos de seção \\ % [cite: 15, 32]
10pt & Resumo, Palavras-chave, Afiliações \\ % [cite: 9, 10, 15]
\hline
\end{tabular}
\end{table}

% Exemplo de Figura [cite: 51, 54]
\begin{figure}[h]
\centering
\includegraphics[width=\columnwidth]{example-image-a} % Substitua pelo arquivo da sua imagem
\caption{Exemplo de uma figura de uma coluna. A resolução deve ser de pelo menos 300 dpi. [cite: 48] Legendas de múltiplas linhas devem ser justificadas. [cite: 53]}
\label{fig:exemplo}
\end{figure}

\lipsum[3-4]

% --- SEÇÕES FINAIS (SEM NUMERAÇÃO) ---
\section*{Agradecimentos}
O cabeçalho da seção de Agradecimentos não deve ser numerado. [cite: 100]

% --- REFERÊNCIAS ---
% O estilo é Vancouver, numerado sequencialmente[cite: 66, 68, 72].
% O tamanho da fonte é 10pt[cite: 65].
\begin{thebibliography}{99}

\bibitem{fogg2003} % Livro [cite: 73, 75]
Fogg BJ. Persuasive technology: using computers to change what we think and do. Boston: Morgan Kaufmann Publishers; 2003. p. 30–5.

\bibitem{hirsh2002} % Artigo de Periódico [cite: 79, 81]
Hirsh H, Coen MH, Mozer MC, Hasha R, Flanagan JL. Room service, AI-style. IEEE Intell Syst. 2002;14(2):8–19.

\bibitem{leclercq2016} % Anais de Conferência [cite: 83, 85]
Leclercq P, Heylighen A. 5.8 analogies per hour: A designer's view on analogical reasoning. In: 7th International Conference on Artificial Intelligence in Design. Dordrecht: Kluwer Academic Publishers; 2016. p. 285–303.

\bibitem{eckes2000} % E-Book [cite: 88, 90]
Eckes T. The developmental social psychology of gender [e-book]. Mahwah (NJ): Lawrence Erlbaum; 2000 [cited 2025 May 27]. Available from: netLibrary e-book.

\bibitem{steiner2013} % Artigo de E-Journal [cite: 93, 95]
Steiner A. Understanding hypertext in the context of reading on the web: Language learners' experience. Curr Issues Educ [Internet]. 2013 [cited 2025 May 27];6(12):2015–219. Available from: http://cie.ed.asu.edu/volume6/number12/

\end{thebibliography}

\end{document}
% =============================================================================
% FIM DO ARQUIVO artigo.tex
% =============================================================================